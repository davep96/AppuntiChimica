\documentclass[../AppuntiChimica]{subfiles}

\begin{document}
	\section{Lezione 5, 21 Marzo 2018}
	Il guscio elettronico più spesso è quello coinvolto nelle reazioni e nel comportamento chimico di un certo elemento. La disposizione degli elettroni più esterni interviene anche nelle proprietà fisiche, per esempio il magnetismo. Anche quelli interni sono importanti per quanto segue. 
	
	Si discute ora di alcune caratteristiche della tavola periodica. Innanzitutto, gli elementi che sono presenti in una determinata sostanza chiica possono essere distinti in categorie:
	\begin{enumerate}
		\item Metalli (Conducono)
		\item Non metalli (non conducono)
		\item Semi-Metalli (semiconduttori) (a volte conducono)
	\end{enumerate}
	Ciascun elemento ha anche altre caratteristiche, come la lucentezza e malleabilità. I semimetalli più noti sono silicio, germanio, arsenico, antimonio, tellurio (tra la 15esima e 16esima colonna). Dalla metà del 1800, molte informazioni erano già state acquisite a riguardo di alcuni elementi come idrogeno, ossigeno e i cosiddetti alogeni (della 17esima colonna). Tutti gli elementi sono suddivisi in 18 gruppi, rappresentati dalle colonne. Perché la tavola è detta periodica? Le proprietà chimiche, si notò, si ripetono in maniera periodica a seconda della massa degli elementi. Da questa periodicità, nacque il nome di tavola periodica. Le righe sono dette periodi. Nelle tavole periodiche ci sono due righe ulteriori, i lantanidi e gli attinidi. Queste appartengono al sesto e settimo periodo rispettivamente. Elementi dello stesso gruppo hanno lo stesso numero di elettroni nel guscio di valenza, mentre elementi nello stesso periodo no. Per questo le proprietà chimiche di elementi dello stesso gruppo sono simili. Il primo gruppo è detto dei metalli alcalini, il secondo è detto dei metalli alcalino-terrosi. Sono dei metalli alcalino terrosi sono meno solubili di quelli alcalini e perciò si trovano molto spesso nei minerali. Per esempio il carbonato di calco ha il calcio, metallo alcalino terroso. Dopodiché ci sono i gruppi tra il 2-12 che sono detti metalli di transizione, che sonno gruppi di raccordo tra il secondo e il tredicesimo. Nel tredicesimo gruppo sono tutti metalli a parte il boro. Nel quattordicesimo, ci sono sia metalli che semi metalli che non metalli. Il sedicesimo gruppo è detto dei calcogeni, hce generano legami con il rame. L'ultimo gruppo è quello dei gas nobili. Questi sono gli elementi che interagiscono di meno perché la loro shell è completa. All'interno dello stesso gruppo, dall'alto verso il basso, si nota un aumento del raggio atomico. Mentre andando da sinistra verso destra nello stesso periodo c'è una diminuzione del raggio. Perché avviene questo? Diciamo prima che per uno ione positivo il raggio è più piccolo, mentre se lo ione è negativo il raggio è maggiore. Vediamo cosa succede in un periodo. Man mano che si va avanti aumenta il numero di elettroni e la carica nucleare. Quindi l'attrazione sugli elettroni più esterni prevale sulla repulsione degli eletttroni sullo stesso guscio, ciò porta ad una riduzione del raggio atomico. La repulsione tra elettroni prende il nome di schermatura. La schermatura da parte di elettroni dello stesso guscio è poco efficace quindi si ha un effetto preponderante del nucleo che tende ad aggreggare maggiormente gli elettroni a se. Nel gruppo invece, il raggio aumenta perché aumenta il numero quantico principale. DUnque, all'interno di un periodo la schermatura è meno efficace perché viene esercitata da elettroni dello stesso guscio. Mentre all'interno dello stesso gruppo la schermatura degli elettroni interni è più efficace e quindi il raggio aumenta. La stessa cosa succede per l'energia di ionizzazione. Questa è l'energia che è necessario fornire ad una specie atomica perché liberi un elettrone. Nell'interno di uno stesso gruppo l'energia di ionizzazione tende a diminuire, perchP c'è un effetto i schermo da parte degli elettroni più interni, mentre all'interno di un periodo l'energia di ionizzazione aumenta. Quindi l'energia di ionizzazione varia in maniera opposta al raggio ma in maniera opposta nel periodo e nel gruppo. Infatti nel periodo aumenta a causa sempre della scarsa schermatura. Si può anche misurare l'effinità elettronica: l'energia rilasciata dall'atomo quando questo acquista un elettrone. Si dice dopo di questo che diventa un anione. Definiamo altre due grandezze riportate per ciascun elemento. Queste sono stato di ossidazione ed elettronegatività. Il concetto di stato di ossidazione è fondamentale perché è utile per:
		\begin{enumerate}
			\item Scrivere la formula chimica 
			\item Bilanciare le reazioni di ossidoriduzione
		\end{enumerate}
	Definiamo prima l'elettronegatività. Questo concetto fu introdotto in molti modi, la versione definitiva è quella di Pauling, che la definì come la misura della tendenza dell'atomo di attrarre a sé elettroni del legame chimico. Il fluoro è l'elemento pi elettronegativo e ha elettronegatività di $ 3.96 $ e il cesio è quello meno elettronegativo, con un'elettronegatività di $ 0.79 $. Da ciò deriva lo stato di ossidazione, che è la carica che assumerebbe quell'elemento in un composto se gli elettroni di legame venissero attribuiti all'atomo più elettromegativo. Vediamo adesso di enumerare le regole in base alle quali si attribuisce il numero di ossidazione. Il numero atomico si scrive al di sopra dell'elemento chimico. Le regole per stabilire il numero di ossidazione sono, in ordine di importanza:
	 
	\begin{enumerate}
		\item  Il numero di ossidazione, di un elemento puro è zero.  Per esempio il numero di ossidazione di $ \chem{N}_{2} $ è zero.
		\item Il numero di ossidazione dell'idrogeno è quasi sempre uno, tranne che negli idruri metallici, in cui è $ -1 $. Per esempio $ \chem{NaH},\ \chem{CaH}_{2} $ sono idruri, quindi il nmumero di ossidazione è $ -1 $.
		\item Il numero di ossidazione dell'ossigeno è $ -2 $ eccetto che nei perossidi e nei peracidi in cui è $ -1 $. Un esempio è l'acqua ossigenata o perossido di idrogeno: $ \chem{H}_{2}\chem{O}_{2} $. In questo caso, dovendosi sommare a zero il numero di ossidazione totale, ed essendo il numero di ossidazione di $ \chem{H}=1 $. In alcuni composti detti superossidi è $ 0.5 $. Alcuni esempi di superossidi sono: $ \chem{NaO}_{2} $ e $ \chem{KO}_{2} $. C'è solo un'eccezione alla regola che è $ \chem{OF}_{2} $.
		\item Il fluoro ha sempre come numero di ossidazione $ -1 $, il cloro ha numero di ossidazione $ -1 $ tranne che nei composti con fluoro oppure ossigeno. Il bromo ha numero di ossidazione -1 tranne quando è combinato con fluoro ossigeno o cloro.
		\item IL numero di ossidazione dei metalli alcalini (del primo gruppo) è +1. I metalli alcalino terrosi, hanno numero di ossidazione +2.
		\item $ \chem{B},\ \chem{Al} $ hanno numero di ossidazione solo +3, l'argento +1, cadmio e zinco +2
		\item Il numero di ossidazione di un catione o di un anione (ioni positivi o negativi) coincide con la carica. 
		\item La somma algebrica di tutti i numeri di ossidazione di un composto deve coincidere con la carica globale del composto. Dunque se il composto è carico è non nullo.
		\item Quando è presente più di un atomo di un elemento in un composto, il numero di ossidazione di quell'elemento in quel composto è il valore medio del numero di ossidazione di quell'elemento in quel composto. 
	\end{enumerate}

	\begin{exe}
		Calcolare, quanti grammi di cloruro di alluminio $ \chem{AlCl}_{3} $ si possono ottenere da $ \SI{2.70}{\gram} $ di alluminio e $ \SI{4.05}{\gram} $ di $ \chem{Cl}_{2} $ nella reazione:
		\begin{gather}
		2\chem{Al}+3\chem{Cl}_{2}\to 2\chem{AlCl}_{3}
		\end{gather}
		Se sono stati ottenuti solo $ \SI{3.50}{\gram} $ di prodotto, qual è stata la resa percentuale?
	\end{exe}
\begin{svol}
	Iniziamo trasformando tutto in moli. Il numero di moli di $ \chem{Al} $ è:
	\begin{gather}
	n_{\chem{Al}}=\dfrac{2.7}{26.982}=\SI{1e-1}{\mole} \\
	n_{\chem{Cl}_{2}}=\SI{5.79e-2}{\mole}
	\end{gather}
		Facendo il calcolo si trova che, con quel numero di moli di alluminio, bisognerebbe, moltiplicando per il rapporto dei coefficienti stechiometrici, una quantità di $ \SI{0.150}{\mole} $ di cloro. Però non si dispone di questa quantità, dunque il cloro è l'agente limitante. Dunque il numero di moli di prodotto che si possono ottenere è di $ \SI{3.80}{\mole}=\SI{5.08}{\gram} $. La resa percentuale è data dal rapporto di resa effettiva e resa teorica moltiplicato per cento. Dunque: $ 68.9\% $.
	
	
\end{svol}
\end{document}
