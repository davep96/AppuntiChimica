\documentclass[../AppuntiChimica]{subfiles}
%
%\documentclass[10pt, a4paper]{article}
%\usepackage[T1]{fontenc}
%\usepackage[utf8]{inputenc}
%\usepackage[italian]{babel}
%
%\usepackage{subfiles}
%\usepackage{amsmath}
%\usepackage{amssymb}
%\usepackage{amsthm}
%\usepackage{mathrsfs}
%\usepackage{physics}
%\usepackage{bbold}
%
%\usepackage{siunitx}
%\usepackage[margin=1.00in]{geometry}
%\usepackage{lmodern}
%
%%Ambienti matematici
%
%\theoremstyle{plain} 
%\newtheorem{teo}{Teorema}[section]
%\newtheorem{cor}{Corollario}[teo]
%\newtheorem{lem}[teo]{Lemma}
%\newtheorem{prop}[teo]{Proposizione} 
%\newtheorem{post}{Postulato}
%\renewcommand\thepost{\Roman{post}}
%
%\newtheorem*{post*}{Postulato}
%
%\theoremstyle{definition} 
%\newtheorem{defn}{Definizione}
%\newtheorem{nb}{Importante}[subsection]
%
%\theoremstyle{remark} 
%\newtheorem*{oss}{Osservazione} 
%\newtheorem{ese}{Esempio}
%
%\newtheorem{exe}{Esercizio}
%\newtheorem*{svol}{Svolgimento}
%
%\newcommand{\scal}[2]{\ensuremath{\left<#1,#2\right>}}
%\newcommand{\com}[2]{\ensuremath{\left[#1,#2\right]}}
%\newcommand{\der}[3][]{\ensuremath{\dfrac{d^{#1}#2}{d#3^{#1}}}}
%\newcommand{\derp}[3][]{\ensuremath{\dfrac{\partial^{#1}#2}{\partial#3^{#1}}}}
%\newcommand{\prob}[1]{\ensuremath{\mathcal{P}\left[#1\right]}}
%\renewcommand{\mod}[1]{\ensuremath{\left|#1\right|}}
%
%\newcommand{\hilb}{\ensuremath{\mathcal{H}}}
%\newcommand{\chem}[1]{\ensuremath{\text{#1}}}

\begin{document}
	\section{Lezione 3, 14 Marzo 2018}
	Nella scorsa lezione abbiamo introdotto il concetto di soluzione. Una soluzion i n generale è una mescolanza di più elementi, dispersi e disperdenti a seconda del rapporto delle quantità. In base alla grandezza delle particelle si classifica l'omogeneità. In particolare noi ci occuperemo di soluzioni, con una singola fase, omogenea.
	
	Si definice concentrazione di un soluto in un solvente la quantità di soluto che è solubilizzata in un a certa quantità di solvente. Si può già intuire l'importanza di conoscere la concentrazione. Si definisce anche la solubilità come la massima concentrazione di soluto che is può avere nel solvente. Oltre a questo livello la soluzione è detta satura. Per concentrazioni maggiori, in condizioni particolari, si può anche parlare di concentrazioni sovra-sature. Queste sono condizioni particolari non ottenibili con condizioni standard (da non confondere con condizioni normali). In altrernativa si ha la precipitazione di parte del soluto nel solvente. Per esempio, nel bollire l'acqua, si può sciogliere il sale fino ad una certa quantità, dopodiché questo si sedimenta sul fondo come corpo di fondo. 
	
	Ci sono due tipi di misura della concentrazione. I primi tipi di misura della concentrazione sono indipendenti dalla temperatura, i secondi ne sono dipendenti. Conseguentemente le seconde possono essere espresse anche in funzione del volume. Nella prima categoria, di misure indipendenti dalla temperatura invece non ci sarà possibilità di esprimere neppure in base al volume.
	
	Nella prima categoria vi sono tre modi di misura:
	\begin{enumerate}
		\item Percentuale in massa 
		\item La frazione molare
		\item La molalità
	\end{enumerate}
	Si noti che la molalità si indica con la $ m $ minuscola (da non confondere con la massa $ m $ che ha lo stesso simbolo).
	
	Viceversa le misure della concentrazione che sono dipendenti dalla temperatura sono:
	\begin{enumerate}
		\item Percetuale massa volume
		\item Percentuale in volume
		\item La molarità
	\end{enumerate}
	
	Definiamo adesso una ad una ciascuna di queste e vediamo quando possono essere utilizzate più efficientemente.
	
	La percentuale in massa espressa anche come $ \%(m/m) $ è definita come segue:
	\begin{gather*}
		\%(m/m):=\dfrac{\text{Massa del soluto}}{\text{Massa del solvente}}\times 100
	\end{gather*}
	questa spesso è utilizzata per soluti in gas. Si caratterizza anche dal fatto che è un numero puro e adimensionale.
	
	La frazione mlare, che si indica con $ X_{soluto} $ è definita come:
	\begin{gather*}
		X_{soluto}:=\dfrac{\text{Numero di moli del soluto}}{\text{Numero di moli del soluto + numero di moli del solvente}}
	\end{gather*}
	è d'uso qui utilizzare il numero tra zero e uno senza moltiplicare per cento. Anche la frazione molare è spesso utilizzata per mescolanze di gas.
	
	Definiamo ora la molalità. Come detto questa è indicata con: $ m $. Indica il numero di moli di soluto in un kilogrammo di solvente.
	\begin{gather*}
		m:=\dfrac{\text{Numero di moli di soluto}}{\text{Kilogrammo di solvente}}
	\end{gather*}
	Si noti che non bisogna prendere un kilogrammo di soluzione, ma un kilogrammo di solvente! Questi sono i tre modi di misura della concentrazione indipendente dal volume o temperatura. Si noti che infatti questi non dipendono in alcun modo dalla temperatura o dal volume nelle loro definizioni.
	
	Il primo tipo di misura della concentrazione della seconda classe è la percentuale in volume. Questa viene spesso scritta come: $ \%(v/v) $. Viene spesso usata per soluti liquidi. La percentuale di volume è definita come:
	\begin{gather*}
		\%(v/v):=\dfrac{\text{Volume del soluto}}{\text{Volume della soluzione}}\times 100
	\end{gather*}
	Si suppone sempre negli esercizi che i volumi siano completamente additivi. Questo sperimentalmente non è vero. Nel caso nostro, in cui i volumi siano additivi il volume della soluzione è uguale alla somma del soluto e del solvente.
	
	Segue la percentuale di massa sul volume. Questa è definita come:
	\begin{gather*}
	\%(m/v):=\dfrac{\text{Massa del soluto in grammi}}{\text{Volume della soluzione in millilitri}}\times 100
	\end{gather*}
	
	L'ultima, quella usata la maggior parte delle volte è la molarità. Questa si indica con $ M $. Questa è definita come il numero di moli di soluto in un litro di soluzione:
	\begin{gather*}
	M:=\dfrac{\text{Numero di moli di soluto}}{\text{Volume di soluto in litri}}
	\end{gather*}
	Una soluzione 1 molare per esempio ha una mole di soluto per litro. Si usa scivere anche:
	\begin{gather*}
		\left[\text{formula chimica}\right]=1M
	\end{gather*}
	per indicare la concentrazione della specie di cui la formula è data tra parentesi quadre. Per passare dal primo al secondo gruppo di misure è necessario conoscere la  densità. Questa infatti fa comparire il volume e quindi può essere usata come link tra le due classi di metodi. 
	
	Una soluzione si può preparare ad una certa concentrazione e portarla ad una concentrazione minore per diluizione. Questo processo si basa sul fatto che aggiungendo del solvente ulteriore cambia la composizione chimica, mantenendo la quantità di soluto costante. Diluire una soluzione significa diminuire la concentrazione. Si basa sul fatto che diluendo la concentrazione diminuisce ma la quantità chimica di soluto, ossia il numero moli di soluto, rimane costante. Questo concetto è molto intuitivo.
	
	Dalla definizione di molarità deduciamo che il numero di moli di soluto si pu esprimere come prodotto tra molarità e volume di soluzione. Siccome il numero di moli di soluto non cambia per diluizione, il numero di moli di soluto sarà uguale anche alla molarità finale per il volume finale dopo la diluizione. Per ciò sappiamo che, per effetto della diluizione, accade che la molarità cambia a:
	\begin{gather*}
	\label{eqn:dilmol}
	M_{f}=\dfrac{V_{i}}{V_{f}}M_{i}
	\end{gather*}
	data la costanza delle moli di soluto, aumentando il volume della soluzione con il solvente la molarità finale risulta minore di qulla iniziale. Si ottiene una soluzione a concentrazione minore di quella iniziale. Per chiudere questa parte possiamo dare questa ultima considerazione che il numero di moli, o la quantità chimica di sostanza, può essere determinata o pesando una quantità di soluto e dividendo per la massa molare di solvente, oppure, nel caso di una soluzione a molarità notà p uguale alla molarità per il volume di soluzione. Quando si voglia conoscere il numero di moli di soluto, o se ne pesa la quantità esatta, dividendo poi per la massa molare, oppure se ne prende una concentrazione a molarità nota e moltiplicare questa per il volume. Negli esercizi il numero di moli si trova sempre in uno di questi due modi. 
	
	\begin{exe}
		Quanti grammi di una soluzione di acido fosforico $ \chem{H}_{3}\chem{PO}_{4} $ (ossia l'acido fosforico è un soluto) all'$ 85\% $ in massa, contengono $ \SI{25}{\gram} $ di acido fosforico puro.
	\end{exe}
	\begin{svol}
		Si ottiene che la massa della soluzione  uguale a:
		\begin{gather*}
		m_{sol}=\dfrac{m_{\chem{H}_{3}\chem{PO}_{4}}}{85}\times 100 = \SI{29.4}{\gram}
		\end{gather*}
		Il risultato è banale perché poi prendendo l'$ 85\% $ di $ 29.4 $ si ottiene $ 25 $.
	\end{svol}
	\begin{exe}
		Un campione di cloruro di sodio $ \chem{NaCl} $ ha una massa di $ \SI{9.65}{\gram} $ viene solubizzato con l'acqua deionizzata e la soluzione risultante ha un volume uguale a $ \SI{125}{\milli\liter} $. Determinare la concentrazione della soluzione espressa come percentualle massa volume e come molarità.
	\end{exe}
	\begin{svol}
		La percentuale massa volume è facilmente calcolabile perché per definizione è:
		\begin{gather*}
		\%(m/v)=\dfrac{m_{soluto}}{V_{soluz}}\times 100=7.72\%
		\end{gather*}
		Questa soluzione dunque ha una concentrazione di cloruro di sodio massa volume del $ 7.72\% $. Si poteva ottenere lo stesso risultato conla seguente proporzione:
		\begin{gather*}
		\dfrac{m_{soluto}}{v_{soluz}}=\dfrac{x}{\SI{100}{\milli\liter}}
		\end{gather*}
		L'utilizzo delle proporzioni si può fare ma bisogna stare attenti a non sbagliare. Per calcolare la molarità è necessario conoscere la massa molare del soluto. Nel caso del cloruro di sodio la massa molare è uguale alla somma del peso atomico del sodio con quello del cloro, e viene $ \SI{58.44}{\gram\per\mole} $. Da cui:
		\begin{gather*}
		n=\dfrac{\SI{9.65}{\gram}}{\SI{58.44}{\gram\per\mole}}=\SI{0.165}{\mole}
		\end{gather*}
		la molarità è dunque il numero  di moli sul volume espresso in litri. Dunque:
		\begin{gather*}
		M=\dfrac{\SI{0.165}{\mole}}{\SI{12.5e-3}{\liter}}=\SI{1.32}{\mole\per\liter}
		\end{gather*}
	\end{svol}
	\begin{exe}
		Supponiamo di avere una soluzione di bromuro di sodio $ \chem{NaBr} $ preparata solubilizzando $ \SI{56.31}{\gram} $ in $ \SI{125.2}{\milli\liter} $ di soluzione (non soluto, soluzione). Calcolarne la molarità. Se si vuole portare la concentrazione ad essere due molare, a quale volume bisogna portare la soluzione? quale volume di acqua bisogna aggiungere? (con la solita supposizione di additività)
	\end{exe}
	\begin{svol}
		Il numero di moli è dato dal rapporto del numero di grammi e la massa molare:
		\begin{gather*}
			n=\dfrac{\SI{56.31}{\gram}}{102.98}=0.5468
		\end{gather*}
		La concentrazione molare è quindi:
		\begin{gather*}
		\left[\chem{NaBr}\right]=\dfrac{0.5468}{\SI{125.2e-3}{\liter}}=\SI{4.367}{\mole}
		\end{gather*}
		Per diminuire la molarità è chiaro che dobbiamo diminuire, dunque dobbiamo utilizzare la formula \ref{eqn:dilmol}. In questo caso va invertita si trova che:
		\begin{gather*}
		V_{f}=\dfrac{M_{i}V_{i}}{M_{f}}=\SI{273.4}{\milli\liter}
		\end{gather*}
		Per ridurre la concentrazione di più della metà abbiamo dovuto aumentare di più del $ 100\% $ il volume. Il volume di acqua che va aggiunta è la differenza dei volumi.
	\end{svol}
	\begin{exe}
		Si ha una soluzione di iodato di sodio: $ \chem{NaIO}_{3} $. Il peso molare del iodato di sodio è:
		\begin{gather*}
			PM_{\chem{NaIO}_{3}}=297.89
		\end{gather*}
		e ioduro di sodio $ \chem{NaI} $ di peso molecolare:
		\begin{gather*}
			PM_{\chem{NaI}}=149.89
		\end{gather*}
		si forma dello iodio $ I_{2} $ e acido solforico $ \chem{H}_{2}\chem{SO}_{4} $. La reazion chimica è quindi:
		\begin{gather*}
			\chem{NaIO}_{3}+5\chem{NaI}+3chem{H}_{2}\chem{SO}_{4} \to 3\chem{I}_{2}+2 \chem{Na}_{2}\chem{SO}_{4}+ 2\chem{H}_{2}\chem{O}
		\end{gather*}
		Calcolare quanto iodio si può ottenere con cento grammi di iodato di sodio e duecento di ioduro di sodio.
	\end{exe}
	\begin{svol}
		Bisogna innanzitutto trovare chi è l'agente limitante. Iniziamo però trasformando tutto in moli. Abbiamo che:
		\begin{gather*}
			n_{\chem{NaIO}_{3}}=\SI{0.505}{\mole}\\
			n_{\chem{NaI}}=\SI{1.334}{\mole}
		\end{gather*}
		si sono tradotti i grammi in moli. Queste sono le moli sperimentalmente a disposizione. Si va a cercare l'agente limitante. Essendo in rapporto uno a cinque, il numero di moli di \chem{NaI} deve essere cinque volte quello di $ \chem{NaIO}_{3} $. Se moltiplico per cinque il numero di moli di $ \chem{NaI} $ trovo $ \SI{2.525}{\mole} $. Questo è maggiore di quanto si ha a disposizione, quindi $ \chem{NaIO}_{3} $ è l'agente limitante. Nei calcoli seguenti si farà sempre riferimento alla quantità di questa specie chimica. Quindi il numero di moli di $ \chem{I}_{2} $ che si ottengono è $ 3/5 $ di quello dell'agente limitante ossia:
		\begin{gather*}
		n_{\chem{I}_{2}}=\dfrac{3}{5}\times\SI{1.334}{\mole}=\SI{0.800}{\mole}
		\end{gather*}
		La massa si ottiene moltiplicando per il coefficiente corretto. Se al problema si fosse aggiunta anche l'informazione che la resa fosse una certa percentuale allora avremmo dovuto moltiplicare questo coefficiente per la resa. Spesso nella realtà la resa effettiva è minore del cento percento.
	\end{svol}
\end{document}
