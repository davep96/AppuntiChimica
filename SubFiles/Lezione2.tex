\documentclass[../AppuntiChimica]{subfiles}

%\usepackage[T1]{fontenc}
%\usepackage[utf8]{inputenc}
%\usepackage[italian]{babel}
%
%\usepackage{subfiles}
%\usepackage{amsmath}
%\usepackage{amssymb}
%\usepackage{amsthm}
%\usepackage{mathrsfs}
%\usepackage{physics}
%\usepackage{bbold}
%
%\usepackage[margin=1.00in]{geometry}
%\usepackage{lmodern}
%
%%Ambienti matematici
%
%\theoremstyle{plain} 
%\newtheorem{teo}{Teorema}[section]
%\newtheorem{cor}{Corollario}[teo]
%\newtheorem{lem}[teo]{Lemma}
%\newtheorem{prop}[teo]{Proposizione} 
%\newtheorem{post}{Postulato}
%\renewcommand\thepost{\Roman{post}}
%
%\newtheorem*{post*}{Postulato}
%
%\theoremstyle{definition} 
%\newtheorem{defn}{Definizione}
%\newtheorem{nb}{Importante}[subsection]
%
%\theoremstyle{remark} 
%\newtheorem*{oss}{Osservazione} 
%\newtheorem{ese}{Esempio}
%
%\newtheorem*{exe}{Esempio}
%\newtheorem*{svol}{Svolgimento}
%
%\newcommand{\scal}[2]{\ensuremath{\left<#1,#2\right>}}
%\newcommand{\com}[2]{\ensuremath{\left[#1,#2\right]}}
%\newcommand{\der}[3][]{\ensuremath{\dfrac{d^{#1}#2}{d#3^{#1}}}}
%\newcommand{\derp}[3][]{\ensuremath{\dfrac{\partial^{#1}#2}{\partial#3^{#1}}}}
%\newcommand{\prob}[1]{\ensuremath{\mathcal{P}\left[#1\right]}}
%\renewcommand{\mod}[1]{\ensuremath{\left|#1\right|}}
%
%\newcommand{\hilb}{\ensuremath{\mathcal{H}}}
%\newcommand{\chem}[1]{\ensuremath{\text{#1}}

\begin{document}
	\section{Lezione 2, 12 Marzo 2018}

	Oggi affrontiamo l'argomento relativo a cosa è una reazione o un'equazione chimica. Introduciamo cosa sia e come venga rappresentata. Una reazione chimica è un processo in cui avviene una trasformazione tra una o più sostanze dette reagenti a una o più sostanze dette prodotti. Si scrive nel seguente modo:
	\begin{align*}
	\text{reagenti}&\to \text{prodotti} \\
	A+B+C+\dots&\to E+F+\dots
	\end{align*}
	I prodotti sono in principio composti chimici completamente differenti. QUali sono le informazioni necessarie da fornire per specificare una reazione chimica. Innanzitutto la prima informazione che è necessario fornire è la natura chimica dei reagenti e dei prodotti. Questo significa che, in una reazione chimica ci venga detto chi sono i reagenti e chi i prodotti. Bisogna specificare sempre la natura chimica dei reagenti e dei prodotti. La seconda informazione basilare è il rapporto tra il numero di molecole partecipanti alla reazione. Questi coefficienti sono detti coefficienti stechiometrici. Rappresentano il numero totale di moli o molecole che partecipano lla reazione:
	\begin{gather*}
	aA+bB+\dots\to cC+dD+\dots
	\end{gather*}
	Bisogna dire in che rapporto i reagenti reagiscono e in che rapporto i prodotti si producono. Non esiste una regola generale per bilanciare una reazione chimica. L'unica legge che bisogna tenere conto nei bilanciamenti è la legge di Lavoiser (di conservazione della massa): il numero totale di atomi di giascun elemento deve essere conservato nella reazione chimica. Questi possono non essere distribuiti nello stesso modo tra i composti, ma il numero totale di atomi di una specie deve essere uguale. I coefficienti stechiometrici devono essere i numeri interi più piccoli possibili. Infatti indicano un rapporto non una quantità fissa. Quando sono stati attribuiti tutti i coefficienti stechiometrici allora si può dire che l'equazione chimica è stata bilanciata. Un altro elemento che deve essere sempre indicato (ma spesso non lo è) è lo stato di aggregazione delle varie sostanze che partecipano alla reazione chimica. Se sono dei solidi, liquidi gas o sostanze in soluzione. Viene indicato a pedice $ s,\ g,\ l,\ q $ per solidi gas liquidi o soluzioni. In alcuni testi questa specificazione si omette perché spesso non è una informazione necessaria. Risulta invece fondamentale per risolvere un determinato problema in modo pratico conoscere se una sostanza è un gas o un solido o un liquido. Nella reazione chimica si indicano anche solamente le sostanze chimiche che cambiano. 
	
	Tutte le reazioni chimiche appartengono ad una delle segueni quattro classi:
	\begin{enumerate}
		\item Acido-Base (in cui vengono scambiati dei protoni)
		\item Reazioni di precipitazione (ha a che fare con la solubilità)
		\item Complessazione o complessamento (portano alla formazione di complessi di composizione, con un metallo pesante al centro e dei leganti organici, costituiti da carbonio e ossigeno che lo circondano).
		\item Ossidoriduzione (in cui c'è un passaggio di elettroni tra una specie ed un'altra).
	\end{enumerate}
	Queste sono le classi di reazioni chimiche, e queste sono rappresentate come scritto sopra.
	
	Facciamo subito un esempio. Tuttavia quanto abbiamo detto è solo teorico, dal punto di vista pratico non è sempre così. Supponiamo di avere:
	\begin{gather*}
	2\chem{Ca}(\chem{N}\chem{O}_{3})_{2}+2\chem{Na}_{3}\chem{P}\chem{O}_{4}\to \chem{Ca}_{3}(\chem{P}\chem{O}_{4})_{2}+6\chem{Na}\chem{N}\chem{O}_{3}
	\end{gather*}
	I primi due e l'ultimo sono in soluione l'altro è solido. Il primo composto si chiama nitrato di calcio. Il secondo si chiama fosfato di sodio. Queste due sostanze chimiche reagiscvono a dare il fosfato di calcio e nitrato di sodio. Questa reazione è bilanciata. Si può leggere a livello micreoscopico, con tre molecole di nitrato di calcio reagiscono con due di fosfato di sodio a dare una di fosfato di calcio e sei di nitrato di sodio. A livello macroscopico tre moli di \dots. Si può anche vedere come rapporti di masse. Conosciuti i numeri di moli uno può conoscere le masse e fare le reazioni. I n questo modo si trova che, in ordine:
	\begin{gather*}
	492.27+327.88\to 310.18+509.97
	\end{gather*}
	Dove tutto è espresso in grammi. La legge di Lavoiser è dunque verificata. Può capitare di trovare reazioni chimiche espresse con coefficienti stechiometrici frazionali. Questo non  rigoso, ed è meglio non usarlo. Bisogna anche citare la legge di avogadro che specifica che: volumi uguali di gas diversi nelle medesime condizioni di temperatura e pressione contengono lo stesso numero di molecole.SI chiamano condizioni normali quelle per cui: la temperatura è zero gradi centigradi e la pressione è un'atmosfera. Il volume per mole viene dunque di $ \SI{22.41}{\liter} $.
	
	Introduciamo una perturbazione all'aspettativa teorica. Risulta più difficile bilanciare una reazione nel caso in cui determinati atomi sono presenti tra i prodotti in più specie chimiche diverse. Le reazioni chimiche si possono anche rappresentare come somma di due reazioni indipendenti che si hciamano semireazioni. DIciamo che il metodo id bilanciamento basato sulla conservaizone della massa è quello più generale piossibile e in generale questa è la legge a cui bisogna fare riferimento. Possiamo dire che una reazione chimica bilanciata esprime i rapporti quantitativi molari secondo cui le varie sostanze rendono parte alla reazione. Ci permette di calcolare la quantità di mat3eria necessaria per trasformare i reagenti din prodotti. Abbiamo visto finora trasformazioni totali per trasformare i reagenti in prodotti. Risulta quindi possibile trovare le masse dei prodotti.
	
	Per esempio:
	\begin{gather*}
	\chem{C}_{s}+\chem{O}_{2g}\to
	CO_{2g}
	\end{gather*}
	 questa è la reazione di combustione del carbonio. Nelle reazioni di combustione si hanno come prodotti della reazione anidride carbonica e/o acqua. Se supponiamo di sapere che la massa di carbonio è di $ \SI{6}{\gram} $ sappiamo di avere:
	 \begin{gather*}
	 n_{c}=\dfrac{\SI{6}{\gram}}{\SI{12}{\gram\mole^{-1}}}=\SI{0.5}{\mole}
	 \end{gather*}
	 Dunque reagiscono anche mezza mole di ossigeno, per formare mezza mole di anidride carbonica. Guardando le masse si può anche verificare che la legge di Lavoiser è soddisfatta. La reazione avviene attraverso quantità stechiometricamente esatte di reagenti. In realtà però non sempre le reazioni chimiche avvengono partendo da quantità stechiometricamente esatte di reagenti. Talvolta capita che non si ha  disposizione una quantità intera di moli di sostanza. Tra tutti i reagenti ce ne è uno che è detto agente limitante che limita la quantità di prodotto che si può ottenere.
	 
	 Tutti i calcoli stechiometrici sono basati sulò riconoscimento dell'elemento limitante. Definiamo anche la resa teorica la quantità di prodotto teorica che si dovrebbe ottenere in base ai coefifcienti stechiometrici. Nella realtà a seconda delle condizioni operative edell'efficienza di una reazione si ha una resa effettiva che si rileva sperimentalmente che è sempre minore di quella teorica. Si definisce quindi una resa percentuale che è uguale al rapporto tra la quantità effettiva di prodotto ottenuto e quella teorica prevista. A questo punto si arriva a fare un esempio che chiede di calcolare per la seguente reazione:
	 \begin{gather*}
	 2\chem{Ca}(\chem{P}\chem{O}_{4})_{2g}+6\chem{Si}\chem{O}_{2s}+10\chem{C}_{s}\to 6\chem{Ca}\chem{Si}\chem{O}_{3s}+\chem{P}_{4s}+10\chem{CO}_{g}
	 \end{gather*}
	 Il testo dice che facciamo reagire $ \SI{250}{\gram} $ di fosfato di calcio con $\SI{50}{\gram} $ di silice. QUanto è la massa di potassio prodotta?
	 La prima cosa da fare è di calcolare il numero di moli che reagiscono Si ha che:
	 \begin{gather*}
		 N_{\chem{Ca}(\chem{P}\chem{O}_{4})_{2g}}=\dfrac{250}{PM_{\chem{Ca}(\chem{P}\chem{O}_{4})_{2g}}}=0.806\text{mol} \\
		 N_{\chem{Si}\chem{O}_{2s}}=\SI{0.832}{\mole}
	 \end{gather*}
	Il numero di moli di $ \chem{Si}\chem{O}_{2s} $ necessario per la reazione è tre volte quello di $ \chem{Ca}(\chem{P}\chem{O}_{4})_{2g} $. Servono quindi , per mole di $ \chem{Ca}(\chem{P}\chem{O}_{4})_{2g} $, $ \SI{2.42}{\mole} $ di silice per reagire in maniera stechiometricamente esatta. Questo non accade perché non si hanno a disposizione tutte quelle moli di silice. Dunque la reazione no avviene in maniera esatta. Quindi $ \chem{SiO}_{2} $ è il reagente limitante. A questo punto, identificato il reagente limitante, le moli di $ \chem{Ca}(\chem{P}\chem{O}_{4})_{2g} $ che reagiscono veramente sono: $ \SI{0.277}{\mole} $. Perciò si ha che $ 0.806-\SI{0.277}{\mole} $ rimangono inalterate nella reazione. Prendiamo ora come riferimento il reagente limitante. Il numero di moli di $ \chem{P}_{4} $ è un sesto di quelli di ilice. Si ottengono quindi: $ \SI{0.189}{\mole} $ di fosforo. Si ha quindi che:
	\begin{gather*}
		m_{\chem{P}_{4}}=n_{\chem{P}_{4}}MM_{\chem{P}_{4}}=\SI{17.2}{\gram}
	\end{gather*}
	
	Facciamo un altro esempio di calcolo dei pesi molecolari. Si calcoli la massa molare, il peso molare e il numero di moli contenute in $ \SI{10}{\gram} $ $ \chem{CO}_{2},\ \chem{Na}_{2}\chem{SO}_{4}\bullet10\chem{H}_{2}\chem{O}$.
	
	Nel primo caso, il peso molecolare è: $ 12.011+2\times15.9994=44.009 $. Quindi la massa molare è: $ \SI{44.009}{\gram\per\mole} $. Dunque in $ \SI{10}{\gram} $ ci sono $ \SI{0.2272}{\mole} $. Il secondo caso è analogo.
	
	Si introduce ora il concetto di soluzione. Si definisce un sistema disperso come la mescolanza o miscela di più componenti. Vi è una componente disperdente, presente in maggiore quantità e uno o più componenti dispersi. Similmente una soluzione è un sistema disperso in cui la componente disperdente è detta solvente mentre quella dispersa è detta soluto. Sono inoltre delle miscele omogenee costituite da una sola fase. In questo caso le particelle presenti hanno dimensioni tipicamente minori di $ \SI{1e-9}{\meter}=\SI{1}{\nano\meter} $. Diciamo che la solubilizzazione è un fenomeno che può essere endotermico o esotermico, se richiede o cede calore. Nei casi in cui non si ottenga una vera e propria soluzione le dimensioni delle molecole sono maggiori di un nanometro. Se il diametro è maggiore di $ \SI{100}{\nano\meter} $ si parla di dispersione grossolana. Che non è una soluzione, è una dispersione. Quando la dimensione è dell'ordine di pochi nanometri allora la dispersione è detta colloidale. In entrambi i casi il sistema è eterogeneo, dunque presenta più fasi. 	
\end{document}
