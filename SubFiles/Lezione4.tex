\documentclass[../AppuntiChimica]{subfiles}


\begin{document}
	\section{Lezione 4, 19 Marzo 2018}
	Nella lezione scorsa ci siamo occupati delle soluzioni, delle misure possibile della concentrazioni e delle diluizioni, ossia di come, aumentando il volume possiamo passare a soluzioni con concentrazioni minore. Oggi ci occuperemo della configurazione elettronica degli elementi, cosicché potremmo commentare la tavola periodica degli elementi.
	
	Vediamo cosa si intende per configurazione elettronica e come questo concetto si è evoluto nella storia. Il concetto di configurazione elettronica è fondamentale perché da esso derivano tutti i legami chimici. Questo concetto deriva dal modello atomico, iniziato nei suoi studi da Rutherford, poi evolutosi nel tempo. Si pensava che gli elettroni presenti nella parte più esterna del nucleo, si trovassero su generiche orbite circolari attorno al nucleo. Questo però non è in accordo con la fisica classica perché un elettrone che giri attorno al nucleo secondo l'elettromagnerismo dovrebbe irraggiare fotoni e quindi perdere energia. Fu poi Bohr, a risolvere questo problema ipotizzando che, l'elettrone, nel suo girare attorno al nucleo, potesse girare solo per raggi prestabiliti con energie fissate. Di qui scese il concetto di quantizzazione. Nel modello di Bohr nel passaggio da un'orbita all'altra cambia l'energia dell'elettrone con conseguente emissione di un fotone per compensare la differenza di energia. Questa energia è data da:
	\begin{gather}
	E=h\nu=Ry\left(\dfrac{1}{n_{1}^{2}}-\dfrac{1}{n_{2}^{2}}\right)
	\end{gather}
	Dove $ Ry $ è la costante di Rydberg. $ n_{1} $ ed $ n_{2} $ sono i numeri quantici associati alle orbite iniziali e finali dell'elettrone. Rydberg fece calcoli teorici per trovare questo risultato mentre Bohr fece verifiche sperimentali. Supponendo che il momento angolare fosse quantizzato, allora si verificava che la teoria coicide con i risultati sperimentali. Per superare il limite di validità per solo atomo monoatomico si passò al modello di cosiddetta meccanica ondulatoria. Il primo passo che diede origine a questo modello fu il principio di indeterminazione di Heisemberg.
\end{document}
